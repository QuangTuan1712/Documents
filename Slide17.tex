\begin{frame}{\makebox[0pt][l]{\includegraphics[scale=0.3]{Logotip.png}}\hspace{6.5cm}\textbf{\textcolor{blue}{А}\textcolor{gray}{нализ свойств меры Хартли}}}
    \fontsize{10pt}{13pt}\selectfont
    \setlength{\parindent}{3ex}
    \noindentЭкспериментатор одновременно подбрасывает монету (М) и кидает игральную кость (К). Какое количество информации содержится в эксперименте (Э)?\\
    \vspace*{2mm}
    \noindent\textbf{\textcolor{Green}{Аддитивность}}:\\
    $i$(Э) = $i$(M) + $i$(K) => $i$(12 исходов) = $i$(2 исхода) + $i$(6 исходов): $log_x$12 = $log_x$2 + $log_x$6

    \noindent\textbf{\textcolor{Green}{Неотрицательность}}: \\
    Функция $log_x$N неотрицательна при любом x>1 и N$\geqslant$1.
 
    \noindent\textbf{\textcolor{Green}{Монотонность}}:\\
    С увеличением р(М) или р(К) функция i(Э) монотонно возрастает.

    \noindent\textbf{\textcolor{Green}{Принцип предопределённости}}:\\
    При наличии всегда только одного исхода (монета и кость с магнитом) количество информации равно нулю: $log_x$1 + $log_x$1 = 0.
    
\end{frame}