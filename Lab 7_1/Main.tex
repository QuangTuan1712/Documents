\newpage
\setcounter{page}{40}
\begin{minipage}[t]{0.48\linewidth}

\noindent\textbf{M234.} \emph{Дан квадрат со стороной 1. От него отсекают четыре уголка $-$ четыре треугольника, у каждого из которых две стороны идут по сторонам квадрата и составляют $1/3$ их длины. С полученных 8-угольником делают то же самое: от каждой вершины отрезают треугольник, две соторны которого составляют по $1/3$ соответствующия сторон 8-угольника, и так далее. Получается последовательность многоугольников (каждый содержится в предыдущем). Найдите площадь фигуры, являющейся пересечением всех этих многоугольников (то есть образованной точками, принадлежащими всем многоугольникам).}

\bigskip

\setlength{\parindent}{7ex}Обозначим через $M_0$ исходный квадрат, через \hspace{2mm}$M_1, M_2, M_3,$ \ldots $-$ многоугольники, получаемое из $M_0$ последовательным отрезанием уголков. Удобно рассмотреть также многоугольник $N_k$, вершинами которого служат середины сторон $M_k (k =  0, 1, 2,$ \ldots).

Пусть $A$ \hspace{0.5mm}$-$ \hspace{1mm}произвольная вершина многоугольника $M_k$ \hspace{1mm}$-$ a \textit{B} и $C$ \hspace{1mm}$-$ соседние с ней вершины. Чтобы получить из многоугольника $M_k$ многоугольник $M_{k + 1}$, нужно от \hspace{2mm}к а ж д о й \hspace{2mm}вершины А многоугольника $M_k$ отрезать треугольник $B_2AC_2$ (см. рис.8) такой, что точки $B_2$ и $C_2$ делят, соответственно, отрезки $[AB]$ и $[AC]$ в отношении 1 : 2. Пусть $B_1$ и $C_1$ середины отрезков и , то есть соседние вершины многоугольника $N_k$; тогда
\begin{align}
    |\hspace{1mm}AC_2\hspace{1mm}| = \frac{\hspace{1.5mm}2\hspace{1.5mm}}{3}\hspace{1mm}|\hspace{1mm}AC_1\hspace{1mm}|, \\
    |\hspace{1mm}AB_2\hspace{1mm}| = \frac{\hspace{1.5mm}2\hspace{1.5mm}}{3}\hspace{1mm}|\hspace{1mm}AB_1\hspace{1mm}|. \nonumber
\end{align}

При переходе от многоугольника $N_k$ к многоугольнику $N_{k + 1}$ точки $B_1$ и $C_1$ остаются его вершинами, но уже не соседними: между ними появляется новая вершина $A_1$ середина отрезка $[B_2C_2]$. Таким образом, многоугольник $N_{k + 1}$ получается из многоугольника $N_k$\hspace{3mm}д о б а в л е н и е м к\hspace{3mm}каждой его стороне $B_1C_1$ треугольника типа $B_1A_1C_1$.
\medskip

\noindent\includegraphics[width = \linewidth]{Picture.png}

\noindentРис. 8.
\end{minipage}\quad
\begin{minipage}[t]{0.45\linewidth}

Из\hspace{2.5mm}(1)\hspace{2.5mm}следует, что площади\hspace{5mm}о т р е з а е м ы х\hspace{5mm}и\hspace{5mm}д о б а в л я е м ы х\hspace{5mm}треугольников связваны такими соотношениями:
\begin{align}
    S\hspace{1mm}(B_2AC_2) = \frac{\hspace{1.5mm}4\hspace{1.5mm}}{9}\hspace{1mm}S\hspace{1mm}(B_1AC_1), \\
    \nonumber\\
    S\hspace{1mm}(B_1A_1C_1) = \frac{\hspace{1.5mm}1\hspace{1.5mm}}{3}\hspace{1mm}(B_1AC_1).
\end{align}
Пусть $x_k$ \hspace{0.5mm}$-$ \hspace{0.5mm}площадь многоугольника $N_k$, $y_k$ \hspace{0.5mm}$-$ \hspace{0.5mm}площадь \hspace{0.5mm}$M_k$. Просуммировав (2) и (3) по всем вершинам \hspace{0.5mm}$A$ \hspace{0.5mm}многоугольника \hspace{1.5mm}$M_k$, получим:
\begin{align}
    y_k - y_{k\hspace{0.5mm}+\hspace{0.5mm}1} = \frac{\hspace{1.5mm}4\hspace{1.5mm}}{9}\hspace{1mm}(y_k - x_k), \\
    \nonumber\\
    x_{k\hspace{0.5mm}+\hspace{0.5mm}1} - x_k = \frac{\hspace{1.5mm}1\hspace{1.5mm}}{3}\hspace{1mm}(y_k - x_k);
\end{align}
отсюда
\begin{align}
    (y_{k\hspace{0.5mm}+\hspace{0.5mm}1} - x_{\hspace{0.5mm}+\hspace{0.5mm}1}) = \frac{\hspace{1.5mm}2\hspace{1.5mm}}{9}\hspace{1mm}(y_k - x_k)
\end{align}
\bigskip
и
\begin{equation}
    \frac{\hspace{1.5mm}y_k - y_{k\hspace{0.5mm}+\hspace{0.5mm}1}\hspace{1.5mm}}{x_{k\hspace{0.5mm}+\hspace{0.5mm}1} - x_k} = \frac{\hspace{1.5mm}4\hspace{1.5mm}}{3}.
\end{equation}

Отметим точки \hspace{1mm}$x_0$, \hspace{1mm}$x_1$, \hspace{1mm}$x_2$, \ldots и \hspace{1mm}$y_0$, \hspace{1mm}$y_1$, \hspace{1mm}$y_2,$ \ldots \hspace{1mm}на числовой оси (рис. 9 и 10). Из соотношения (6) следует, что длина отрезка $[x_{k + 1}$, \hspace{1mm}$y_{k + 1}]$ в $4$ $\frac{1}{2}$ раза меньше длины отрезка [$x_k$, \hspace{1mm}$y_k$]. Рассмотрим точку  , делящую отрезок  в отношении $3:4$ ; тогда
\begin{equation}
    \frac{\hspace{1.5mm}y_k - a\hspace{1.5mm}}{a - x_k} = \frac{\hspace{1.5mm}4\hspace{1.5mm}}{3}.
\end{equation}

Учитывая\hspace{2mm}(7)\hspace{2mm}и\hspace{2mm}(8), получим, что
\begin{equation}
    \frac{\hspace{1.5mm}y_{k\hspace{0.5mm}+\hspace{0.5mm}1} - a\hspace{1.5mm}}{a - x_{k\hspace{0.5mm}+\hspace{0.5mm}1}} = \frac{\hspace{1.5mm}4\hspace{1.5mm}}{3};
\end{equation}
значит, точка $a$ делит в том же отношения $3:4$ и отрезок
\begin{equation*}
    [x_{k + 1}, \hspace{1mm}y_{k+1}].
\end{equation*}

Возьмем самый первый отрезок последовательности: $[x_k$, \hspace{1mm}$y_k] = [\frac{1}{2},1]$; в отношении 3 : 4 
его делит точка $a$ , такая, что $\frac{1 - a}{a - \frac{1}{2}} = \frac{4}{3}$, то есть?
\begin{equation}
    a = \frac{\hspace{1.5mm}5\hspace{1.5mm}}{7}.
\end{equation}

В силу сказанного, эта точка делит и отношении $3:4$ все отрезки $[x_k$, \hspace{1mm}$y_k]$ \hspace{1mm}$(k = 0$, \hspace{1mm}$1$, \hspace{1mm}$2$,\hspace{1mm}\ldots); значит, каждый следующий отрезок последовательности $[x_k$, \hspace{1mm}$y_k]$ полу-
\end{minipage}

%%------------------------------------------------------------------------------------------------------------------

\newpage
\setcounter{page}{16}
\begin{minipage}[ht]{0.48\linewidth}
\begin{tabular}{|c|cc|cc|cc|cc|}
 \hline
 \rowcolor{cyan} \multirow{2}{*}{} & \multicolumn{8}{|c|}{Скупщики}\\
 \cline{2-9}\cline{2-9} \rowcolor{cyan}Склады&\multicolumn{2}{|c|}{1} & \multicolumn{2}{|c|}{2} & \multicolumn{2}{|c|}{3} & \multicolumn{2}{|c|}{4}\\
 \hline
 \multirow{2}{*}{1} & &\textcolor{pink}{80} & & \textcolor{pink}{120} & & \textcolor{pink}{150} & & \textcolor{pink}{50} \\
 & 35 & & & & & & 40 & \\
 \hline
 \multirow{2}{*}{2}& & \textcolor{pink}{60} & & \textcolor{pink}{70} & & \textcolor{pink}{90} & & \textcolor{pink}{120} \\
 & 25 & & 10 & & 40 & & & \\
 \hline
 \multirow{2}{*}{3}& & \textcolor{pink}{120} & & \textcolor{pink}{50} & & \textcolor{pink}{110} & & \textcolor{pink}{100} \\
 & & & 50 & & & & & \\
 \hline
\end{tabular}


\medskip
\noindentТаблица. 6.

\bigskip

\noindentВ клетках, которые не вошли в цикл, всё осталось по-старому.

\setlength{\parindent}{3ex}$-$ 1400 долларов $-$ кругленькая сумма! Давай проверять другие пустые клетки. Может набредём на маршрут, который тоже стоит использовать. Вот, например, начнём с клетки (1,2). Для неё расходы изменятся на
\begin{equation*}
    120+60-70-80=30>0
\end{equation*}
\noindentТысяча чертей! Маршрут (1,2) использовать не стоит. А, может быть, воспользоваться\ldots

$-$ Не трудись, Джо. Я уже проверял: больше из этого плана не выжмет ни доллара сам Данциг.

$-$ Данциг, Данциг\ldots это не тот ли, который обчистил «Бэнк оф\ldots»?

$-$ Нет, Джо, он не из наших. Это тот малый, который придумал этот метод. Правда, ещё до него какие-то красные\ldots
\end{minipage}\quad
\begin{minipage}[ht]{0.45\linewidth}
\noindentзазвонил телефон. Клифф снял трубку, подслушал и закричал:

$-$ Сержант, отставить! Оцепить научную библиотеку штата! Мне $-$ машину и набор наручников!

\bigskip

\noindent\textbf{Немного теории}

\medskip

\noindentЧто же позволило сэкономить на транспортных расходах 1400 долларов? Проследим за действиями ловких гангстеров. Сначала Бэйт нашёл допустимый план перевозок. Метод, которым он при этом воспользовался называется \hspace{2mm} \textit{методом минимального элемента} \hspace{2mm} и понятно почему: в нём перевозки всё время ставятся на маршруты с минимальными тарифами, а если будут два маршрута с одинаковым тарифом, то предпочтение, естественно, нужно отдать тому из них, для которого возможная перевозка больше.

Получив допустимый план, Бэйт и Джо стали пытаться улучшить его \textit{распределительным методом}. Это, пожалуй, самый простой, хотя и не самый быстрый способ улучшения плана перевозок. Но прежде чем излагать этот метод в общем виде, сформулируем строго транспортную задачу \textit{линейного программирования}.

Пусть имеется $m$ поставщиков (складов) и $n$ потребителей, $a_i$ \hspace{1mm}$-$ емкость \textit{i}-го склада, а $b_j$ \hspace{1mm}$-$ потребность \textit{j}-го потребителя. Пусть $x_{i j}$ \hspace{1mm}$-$ перевозка от \textit{i}-го поставщика к \textit{j}-му потребителю. Допустимы только такие планы перевозок, для которых
\end{minipage}