\begin{frame}{\makebox[0pt][l]{\includegraphics[scale=0.3]{Logotip.png}}\hspace{5cm}\textbf{\textcolor{blue}{Н}\textcolor{gray}{естрогий вывод формулы Шеннона}}}

    \fontsize{9pt}{11pt}\selectfont
    \setlength{\parindent}{3ex}
    
    \noindent\textbf{\textcolor{Green}{Задача}.} Монета имеет смещённый центр тяжести. Вероятность выпадения «орла» – 0,25, вероятность выпадения «решки» – 0,75. Какое количество информации содержится в одном подбрасывании?
    
    \bigskip
    
    \noindent\textbf{\textcolor{Green}{Решение}}
    
    \begin{itemize}
        \item Пусть монета была подброшена N раз (N$\to\infty$), из которых «решка» выпала M раз, «орёл» — K раз (очевидно, что N = M + K).
        \item Количество информации в N подбрасываниях: $i_N$ = M * i(«решка») + K*i(«орёл»).
        \item Тогда среднее количество информации в одном подбрасывании:\\$i_1$ = $i_N$/N = (M/N) * i(«решка») + (K/N) * i(«орёл») = p(«решка») * i(«решка») + p(«орёл») * i(«орёл»).
        \item Подставив формулу Шеннона для $i$, окончательно получим:\\
        $i_1$ = -p(«решка») * $\log_x p$(«решка») - p(«орёл») * $log_x p$(«орёл») $\approx$ 0,8 бит.
    \end{itemize}
    
\end{frame}